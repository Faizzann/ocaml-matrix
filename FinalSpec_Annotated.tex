\documentclass[letterpaper,11pt]{article}
% Change the header if you're going to change this.

% Possible packages - uncomment to use
%\usepackage{amsmath}       % Needed for math stuff
%\usepackage{lastpage}      % Finds last page
%\usepackage{amssymb}       % Needed for some math symbols
%\usepackage{graphicx}      % Needed for graphics
\usepackage[usenames,dvipsnames]{xcolor}        % Needed for graphics and color
%\usepackage{setspace}       % Needed to set single/double spacing
\usepackage{hyperref}

%Suggested by TeXworks
\usepackage[utf8]{inputenc} % set input encoding (not needed with XeLaTeX)
\usepackage{verbatim} % adds environment for commenting out blocks of text & for better verbatim
 
% Sets the page margins to 1 inch each side
\usepackage[margin=1in]{geometry}

% Uncomment this section if you wish to have a header.
\usepackage{fancyhdr} 
\pagestyle{fancy} 
\renewcommand{\headrulewidth}{0.5pt} % customise the layout... 
\lhead{CS51 Final Spec} \chead{} \rhead{May 5th, 2013} 
\lfoot{} \cfoot{\thepage} \rfoot{}

\frenchspacing
\setlength\parindent{0pt}
\setlength\parskip{2ex}

\newcommand{\annot}[1]{\textbf{\color{BrickRed} [#1]}}

\begin{document}
\title{CS 51 Final Project Technical Specification: Matrix Module and Simplex Algorithm---{\annot{Annotated!}}}
\author{
Luis Perez $|$ luisperez@college.harvard.edu \\ 
Andy Shi $|$ andyshi@college.harvard.edu \\ 
Zihao Wang $|$ zihaowang01@college.harvard.edu \\ 
Ding Zhou $|$ dzhou@college.harvard.edu
}
\date{May 5, 2013}
\maketitle

%Put your stuff here
{\annot{Note for the reader: This is the annotated version of our CS 51 
technical project specification. All annotations will appear in this color. }}.

\section{Overview}

We have decided to reduce the amount of functions available to the outside
world. The matrix module will still contain the same functionality as before,
but only the functions necessary for the implementation of the Simplex algorithm
will be exposed (row-reduction, for example{\annot{Row Reduction ended up not
being necessary. We have instead exposed a large set of library functions and,
additionally, some functions specific to Simplex that would not be exposed otherwise.
All of this is documented in MatrixI.ml}}). The main focus of the project has
now shifted to having a working simplex algorithm. Other functions, such as
finding the eigenvalues of a matrix, while interesting, have been placed in our
“Cool Features” category. Furthermore, following the idea that Simplex is our
main focus, additional functionality such as parsing input, providing arbitrary
float precision and/or providing a matrix implementation with bignums has been
degraded in importance. All of our attention will be focused on creating a
working simplex algorithm. {\annot{All of this happened, even the parsing input part. 
We were able to parse inputs from plaintext files}}.

\section{Detailed Description}
\subsection{Signatures and Interfaces}
\subsubsection{Matrices}

Our matrix module will actually be a functor that takes in an
\verb@ORDERED_AND_OPERATIONAL@ module. The \verb@ORDERED_AND_OPERATIONAL@ module
will be defined as is below. There will be a global order for the elements, also
as defined below. In our beta implementations, we plan to simply use Ocaml
floats as the elements, but, if time allows, the plan is to make the simplex
algorithm rely on abstract as matrices a possible so that we can change the
\verb@ORDERED_AND_OPERATIONAL@ to something more efficient and/or accurate.
{\annot{In the end, we ended up using the Nums OCaml library in order to obtain
arbitrary precison and size with our elements. We made this decision late in the
coding, but because we coded everything abstractly, changing from Floats to Nums
was as simple as exposing the right files and setting the right modules}}

{\annot{Updated version of this can be found in EltsI.ml}}
\begin{verbatim}
type order = Equal | Less | Greater

module type ORDERED_AND_OPERATIONAL
sig
type t
val zero : t
val one: t
val compare : t -> t -> order

(* Converts a t to a string *)
val to_string : t -> string

val add: t -> t -> t
val subtract: t -> t -> t
val multiply: t -> t -> t
val divide: t -> t -> t

(* For testing *)

(* Prints a t *)
val print: t -> unit

(* Generates the same t each time when called *)
val generate: unit -> t

(* Generates a t greater than the argument passed in *)
val generate_gt: t -> unit -> t

(* Generates a t less than the argument passed in *)
val generate_lt: t -> unit -> t

(* Generates a t in between the two arguments. Returns none if none exist *)
val generate_between: t -> t -> unit -> t option
end
\end{verbatim}

{\annot{A somewhat important change to this section was the addition of a 
generate\_x function. Our abstract elements became too abstract and we
couldn't efficiently generate numbers for testing, so we wrote this function which
basically performs a casting operation. In our actual (non-testing) code, we
wrote a from\_string function which we ended up using more}}.

The functions in the module, since they will be dedicated to matrices and
optimized for matrix elements, will not be exposed to the outside world. The
module will be passed in as a constructor to the Matrix module (defined later),
which will exposed only that while is, once again, necessary for the Simplex
module. Additionally, the \verb@ORDERED_AND_OPERATIONAL@ module will more than
likely have additional values and functions not exposed in its signature. For
example, we plan to define the “epsilon for comparison” for floats inside the
module, but we keep it hidden simply because it's not necessary anywhere else.
{\annot{This was mostly implemented as described.}}

The following signature is the one for the Matrix module, and it also the one
that has changed the most. Most of the functions we originally had were not
necessary for the simplex algorithm, so we have made them ``cool features''
which we will attempt to implement if we have time.

{\annot{This Module changed significantly. For an updated look, see MatrixI.ml}}
\begin{verbatim}
module type MATRIX =
sig
exception NonSquare
exception ImproperDimension

type elt
type matrix
(* represented using Ocaml's built-in Arrays *)

(* empty matrix *)
val empty

(* returns the nth row of a matrix as a matrix*)
val get_row : int -> matrix
(* implemented in terms of get_elt *)

(* returns the nth column of a matrix as a matrix *)
val get_coln : int -> matrix
(*also implemented in terms of get_elt *)

(* returns the element in the nth row and mth column *)
val get_elt : int -> int -> matrix 
(* will just get the (n,m) index of an array *)

(* Takes a list of lists and converts that to a matrix *)
val from_list : (elt list list) -> matrix
(* Will implement using nested match statements *)

(* Scales every element in the matrix by another elt *)
val scale : matrix -> elt -> matrix
(* Will implement by iterating through the matrix and scaling each element *)

(* Adds two matrices. They must have the same dimensions *)
val add : matrix -> matrix -> matrix
(* Will add the elements elementwise and construct a new matrix *)

(* Multiplies two matrices. If the matrices have dimensions m x n and p x q, n
* and p must be equal, and the resulting matrix will have dimension m x q *)
val mult: matrix -> matrix -> matrix

(* Returns the row reduced form of a matrix *)
val row_reduce: matrix -> matrix
(* We will implement the algorithm found in the link above *)

(* Prints out the contents of a matrix *)
print: matrix -> unit
(* Iterate through the matrix and print each element *)
end
\end{verbatim}

The exceptions exists in case the user tries to perform operations which require
a square matrix or matrices with certain dimensions. Matrices have their own
type, and they contain elements of the type \verb@elt@. The following is a list
of helper functions in the module, but not exposed to the outside world.

\begin{verbatim}
(* Will take the dot product of the nth row of the first matrix and the jth
 * column of the second matrix to create the n,j th entry of the resultant 
 * where the matrices are a single dimensional arrays *)
val dot -> matrix -> matrix
(* Will implement this based on the specification in the Algorithms book *)
\end{verbatim}

{\annot{We had \emph{a lot} more helper functions to get a row or column, set a
row or column, perform basic row operations, etc. They all made our lives easier
at some point in the project}}.

The following are cool features which we will try to implement if we have time. 

\textbf{Cool Features}:

\begin{verbatim}
(* Returns the inverse of a matrix *)
val inverse: matrix -> matrix

(* Returns the norm of the matrix *)
val norm: matrix -> elt

(* Transposes a matrix. If the input has dimensions m x n, the output will
 * have dimensions n x m *)
val transpose: matrix -> matrix
(* Will basically ``flip'' the indices of the input matrix

(* Returns the trace of the matrix *)
val trace: matrix -> elt
(* Will check if the matrix is square, then sum up all the elements along its
 * diagonal *)

(* Returns the determinant of the matrix *)
val det: matrix -> elt
(* Will implement this algorithm based on a description in Hubbard. Involves
 * column reducing the input (or row-reducing the transpose) and then keeping
 * track of the operations to build a sequence of coefficients to multiply *)
{\annot{We were able to implement this in time, using LU decomposition}}

(* Returns a list of eigenvalues and eigenvectors of a matrix *)
val eigen: matrix -> (elt *matrix) list option
(* Calculates successive powers of the input matrix, each multiplied by the
* same basis vector. Generates a polynomial and solves for zeros, which
* yields eigenvalues. Repeat for all basis vectors *) 
\end{verbatim}

{\annot{We were able to implement inverse, transpose, trace, and determinant.
The first 3 functions we were able to implement as specified, but the
determinant was implemented by decomposing it into upper and lower triangular
matrices. The functions that weren't implemented were mostly left behind because
they were not essential to the Simplex Algorithm}}

\subsubsection{Simplex Algorithm}

The simplex solves linear programs, which are basically linear inequalities.
Given inequalities, using the simplex algorithm, we can find the maximum value
of a given expression. What the simplex algorithm does is that it takes in
several linear inequalities and writes them so that they are equalities. For
instance $5x + 7y \leq 10$ would be $5x + 7y + s = 10$ where $s \geq 0$ is the
slack variable. Then it takes all these equations and puts them a a matrix where
each entry represents the coefficient of the variable in the expression. The
first row represents the equation we are optimizing. The final column is the
constant values of each equation. Then we will do a series of computations so
that the top row has all nonnegative values. This can be done using the
functions from the Matrix module. 

The steps are as follows:

Look along the first row and find the smallest negative element. Look at the
column of that entry and compute the value/element of each row. If any element
is zero, we can ignore it. Otherwise take the element with the smallest
value/element and make this the pivot point. Reduce the column so that the pivot
point in 1 and any other entry in the column is zero. Repeat the process again
until the top row consists of only nonnegative values. The first entry in the
last column is the maximum. In addition, for any column consisting of all zeroes
and a one, we can find the value of the corresponding variable by looking at the
value of the row containing the one. If the column doesn’t consists of zeroes
and a one, then the value of the corresponding variable is zero. 

This algorithm also works with variables that are negative, inequalities where
the unknowns is greater than a constant, and minimum of the equation. This can
be easily done by either multiplying the whole inequality by -1 or by negating
the variable. If calculating the minimum or negating the variable, we can revert
them later.  

\begin{verbatim}
module type simplex
sig
type elt
type linear_equation
type constraint

val pivot : matrix -> matrix

(* Will implement these if we have time *)
val constraint_from_string : string -> constraint
val linear_from_string : string -> constraint
end
\end{verbatim}

These are helper functions that we think will be used:

\verb@no_neg : matrix -> bool@ 

Check the top row has no negative elements, which can be done by using the next
helper function

\verb@min_of_row : row -> elt*(elt ref)@

A helper for the row reduction which will find the minimum element and its
location.

{\annot{We again had a ton more helper functions, plus we did not anticipate
some edge cases in the simplex algorithm. One such case was the fact that our
algorithm relied on guessing an initial solution to the optimization problem
which works for a majority of cases. However, if the constraint equations are
set up correctly, our initial solution would fail to satisfy the constraint
equations. We then had to follow a complex procedure outlined in the
\emph{Algorithms} book to resolve this}}.

\subsection{Modules and Actual Code}
Please see our attached files or our repo at 
\url{www.github.com/Fantastic-Four/ocaml-matrix}. Files of note include Elts.ml
and Matrix.ml. 

\subsection{Timeline}
\begin{itemize}
\item Week 1 (4/7--4/13):
  \begin{itemize}
    \item Signature definitions 
    \item Setting up Github (everyone on the same page)
    \item Writing makefile
    \item Establishing work distribution
  \end{itemize}

\item Week 2 (4/14--4/20):
  \begin{itemize}
    \item Basic matrix functionality
    \item Some rudimentary form of matrix multiplication and row reduction
    \item Finishing up the element types
  \end{itemize}

\item Week 3 (4/21--4/27):
  \begin{itemize}
    \item Beta testing the simplex algorithm
    \item Implementation of simple input parser for testing (we will most likely
be inputting arrays)
    \item Completion of the matrix module algorithms
  \end{itemize}

\item Week 4 (4/28--5/4)
  \begin{itemize}
    \item Fully working simplex algorithm
    \item Implement a better matrix multiplication algorithm
    \item Add some of the cool features!
  \end{itemize}
\end{itemize}

\subsection{Progress Report}
We have already implemented the design structure for all of the files
(specifically, each file has a signature detailing what it needs to do). The
Makefile is also ready, while rudimentary functionality has been implemented in
the matrix module allowing most of the functions we want matrices to have to be
available already. For evidence of our progress please look at Elts.ml and
Matrix.ml.

\section{Version Control}
We are currently using github for version control. Our repository can be found
here, under the organization Fantastic Four
(\url{www.github.com/Fantastic-Four/ocaml-matrix}).We had some difficulties at
first getting the repo set up on everyone’s computers but we fixed that over the
weekend and we should be good to go.

\end{document}
