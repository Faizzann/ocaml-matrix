\documentclass[letterpaper,12pt]{article}
% Change the header if you're going to change this.

% Possible packages - uncomment to use
%\usepackage{amsmath}       % Needed for math stuff
%\usepackage{lastpage}      % Finds last 
%\usepackage{amssymb}       % Needed for some math symbols
%\usepackage{graphicx}      % Needed for graphics
\usepackage[usenames,dvipsnames]{xcolor}        % Needed for graphics and color
%\usepackage{setspace}       % Needed to set single/double spacing
\usepackage{hyperref}

%Suggested by TeXworks
\usepackage[utf8]{inputenc} % set input encoding (not needed with XeLaTeX)
\usepackage{verbatim} % adds environment for commenting out blocks of text & for better verbatim
 
% Sets the page margins to 1 inch each side
\usepackage[margin=1in]{geometry}

% Uncomment this section if you wish to have a header.
\usepackage{fancyhdr} 
\pagestyle{fancy} 
\renewcommand{\headrulewidth}{0.5pt} % customise the layout... 
\lhead{CS51 Final Proect Report} \chead{} \rhead{May 5, 2013} 
\lfoot{} \cfoot{\thepage} \rfoot{}

\frenchspacing
\setlength\parindent{0pt}
\setlength\parskip{2ex}

\newcommand{\annot}[1]{\textbf{\color{BrickRed} [#1]}}

\newcommand{\vect}[1]{\mathbf{#1}}

\begin{document}
\title{CS 51 Final Project Report}
\author{
Luis Perez $|$ luisperez@college.harvard.edu \\ 
Andy Shi $|$ andyshi@college.harvard.edu \\ 
Zihao Wang $|$ zihaowang01@college.harvard.edu \\ 
Ding Zhou $|$ dzhou@college.harvard.edu
}
\date{May 5, 2013}
\maketitle

Link to demo video: \url{}

\section{Overview}

In the real world, people love to make money. Even in virtual worlds (bitcoin?),
people love to make money. Basically, many problems in business and even
everyday life (but especially business), boil down to maximizing profit or
minimizing costs. Of course, when you speak of minimization or maximization
there are always constraints. Resources are finite and time is precious.
Therefore, in an abstract sense, the problem boils down to minimizing or
maximizing some quantity under certain constraints. For example, you might want
to maximize the number of textbooks $x,y,z$ you buy under the constraints that you
can't buy more than five from one company, the total cost can't be more
than \$500, and the last two books you buy must cost more than \$75. This problem
is precisely the type of problem that our program can now solve. How did we do
this?

To start off, we created a framework. We implemented a matrix module with
standard matrix operations such as multiplication, addition, map (where scaling
follows) as well as many extra features such as row reduction, determinant,
trace, and many more. The matrix module basically allows a user to do almost
anything imaginable with a matrix. In addition, we provided read and write
functionality for easy loading of data, and printing functionality for
straightforward viewing of the abstract matrix. However the intellectually
challenging part of our project is the implementation of the simplex algorithm.
We needed to create a matrix module simply because OCaml doesn't provide one
(also because we just wanted to implement a matrix module, of course). We use
many of the functions in the matrix module to create and manipulate simplex
systems within our simplex module. This allowed us to abstract away yet another
part of our project.

The simplex algorithm is a type of linear programming that allows one to
optimize a linear equation given linear constraints (like the example problem at
the beginning). In mathematical terms, the simplex program operates on linear
programs in standard form. That is, it seeks to minimize (or maximize)
$C\vect{x}$ subject to $A\vect{x} = \vect{b} $ (Wikipedia). $\vect{x}$ and
$\vect{b}$ are vectors. $x$ contains the variables in the equation we want to
optimize (the objective), and $b$ contains the constants in the constraints. $C$
and $A$ are matrices, and they represent the coefficients in the objective and
the constraints, respectively. Let's consider the inequalities: $30x + 15y + 18z
\geq 500, x \geq 5,y \geq 5, z \geq 5$, and $15y+18z \geq 75$ and the objective
functions $x +y + z$ where $x, y, z \geq 0$. As you will notice, this is
precisely the system of equations necessary to solve our book example above. The
algorithm converts the inequalities into equalities by adding positive slack
variables. Then it represents this set of equations in the matrix form where the
coefficients of the terms we maximize are on the first row and each column
represents a variable. This provides affordable storage of most of the
information necessary to solve the linear program.  Furthermore, we then define
basic and non-basic variables (kept track of in a list), which together with the
aforementioned matrix, create a simplex system.

The first function necessary for simplex is called \verb@simple_solve@. This
function, in abstract terms, finds the minimum value of a passed in system (as
described above). When we call the function \verb@simple_solve@, the program
finds the entering and leaving variables which are the variable that would
exchange position from basic and non-basic (slack to non-slack in the initial
case). Here, the algorithm is making the assumption that the system has a
feasible solution where all of our non-basic variables are zero, and therefore
exchanges the role of one of these variables with the corresponding basic
variable. In geometric terms, we are moving along one edge of the n-gon formed
by the constraints and attempting to minimize. This step is accomplished by the
helper function pivot. The function \verb@simple_solve@ continues to loop
(moving along edges and minimizing) until the first row---which represents the
term we want to optimize consists of only non-positive values or there is no
possible entering variable (ie, any change in the variables will only increase
our answer, not minimize it). In the first case, algebraically, it means that
the constraint can only increase if the variables change. Therefore, we have
found our minimum value which is the constant stored at the end of the first
row. This is the optimized value of the objective function. In the second case,
where no suitable entering variable can be found, it means that the system can
be decreased infinitely as pleased and therefore it is Unbounded. In the case
where a solution exists, one possible value for the original variables can also
be found by looking at their corresponding columns and checking to see if they
are basic. If there are an infinite number of these, our algorithm returns only
one. 

What if we had the constraint $x + y + z \geq -10$? As stated above, our
\verb@simple_solve@ makes the assumption that one of the feasible solutions is
the zero solution, but if we initially set $x, y$, and $z$ to 0 (our algorithm
generates a ``basic solution'' by doing this step) the constraint does not hold.
We therefore have a helper function called \verb@initialize_simplex@ that
converts everything into the ``simple'' form so we can just recursively call
\verb@simple_solve@. It does this by following a very complex algorithm better
detailed below. Our final  simplex algorithm now accounts for all cases as long
as the equations are linear. It also handles not only less than or equal
constraints, but also greater than or equal to and equality constraints.
Finally, not only can it minimize, but it can also maximize. You can follow the
algorithm at each step on www.simplexme.com (we used this to debug parts of our
code), but we found that SimplexMe only works for the ``simple'' case. It has
errors for large matrices and the infeasible and unbounded cases. 
	
\section{Planning}
Here is our \href{./DraftSpec_Annotated.pdf}{functional spec} and our
\href{./FinalSpec_Annotated.pdf}{technical spec} (the links open new PDFs). 

We planned to have a completed Simplex solver for simple cases, handling less
than or equal constraints in addition to a complete Matrix module. Here are our
initial functional spec and technical spec. We achieved far more than we
expected to achieve. 

Our matrix module is fairly robust and it certainly implements everything that
is necessary to program the simplex algorithm (which was our main goal for this
module). Yet, it goes further. We managed to implement determinant, trace,
row-reduction,inverse, LU decomposition, and more (all extra functions that we
classified under our ``cool features'' category. There were only minor points
that we did not achieve, such as the eigen function, which finds eigenvectors
and eigenvalues.

As to the Elts module, we really went far and beyond. Because we coded
everything from an abstract point of view, once we had completed our project we
were able to go back and add a new Nums module which provided arbitrary
precision and size (ie, implementing arbitrary float arithmetic and bignum
arithmetic). Of course, to be honest, we really didn't actually code most of
this up. We used the built in Nums library. But as can be seen inside of
Elts.ml, our code was planned out so well that we can replace the foundation of
the mathematics in our program (\verb@ORDERED_AND_OPERATIONAL@) with only a few
changes in code.

The Simplex module also went further than expected (though it certainly was more
difficult to implement than we expected). Not only did we implement the ability
to solve simple linear programs, but we can now solve with certainty any linear
program (dual and primal---dual requires \verb@initialize_simplex@ while primal
is the ``simple'' case), and provide the user with not only the optimal value,
but also one possible solution (which, in most cases is the only solution). 

Most of the milestones were achieved for all of our individual modules, with the
exception of our \verb@row_reduce@ (which was buggy and took longer than
expected), and of our simplex module (we went a day over what we had originally
planned). 

\section{Design and Implementation}

Implementing simplex was difficult. A lot of difficulty stemmed from the
complexity and size of the algorithm itself, and we spent a lot of time trying
to understand each step. We never completely internalized every step of the
algorithm which made debugging difficult because it took a lot of effort to
recall the correct steps. We also had some initial terminology confused which
caused some of the code to be written incorrectly. Since each step ended up
being so complex, whenever we moved to the following steps of the algorithm we
always had trouble recalling what exactly we were doing and if someone else had
already done it. Furthermore, bugs were very difficult to find because the
system was so complex and there weren't many cases for which we could work
out the steps by hand. 

Finding SimplexMe was essential to finally getting all of the bugs out of our
system, but it also added many hours of unnecessary debugging. SimplexMe has
implemented simplex incorrectly and therefore the answer it gives for non-simple
systems can be, at times, incorrect. We verified this with Mathematica's
built in Linear Programming function.

Testing the system also proved to be a problem in its own right. Even after we
got all of the parsing functions down and were certain that the data was being
loaded correctly, we were never really able to find a large set of simplex
example problems. In the end, we resorted to each team member individually
finding a problem and solution (with Mathematica, of course, since SimplexMe
proved unreliable), and then typing out the data into a text file so we could
test it. Once we had all the data, we decided to program an interface that would
allow for expanding the testability of the code. More information on this can be
found in the \verb@README@.

To use the simplex algorithm, we first parse a standardized plain text file
containing the parameters of the objective and the constraints. The first line
of this file must either be \verb@max@ or \verb@min@, signalling a maximization
or minimization step. The next line denotes the objective function. The
coefficients of each of the variables in the objective are separated by columns
(decimals must be represented by fractions because we are using Ocaml's num
library). The last element in this is the ``constant'' (5 if the objective is
$2x + 3y + 5$). The next line must read \verb@subject to@ to help us with
parsing. The next lines are the constraints, where the coefficient of each
variable that appears in the objective must appear (in the same order as in the
objective), separated by commas. We also allow the user to enter if the
constraint is a $\geq, \leq$, or $=$ condition. Again, the last element is the
constant (5 in the case of $x + y \leq 5$). It is assumed that all the variables
in the objective function are $\geq 0$. 

Our program will parse this text file and generate a matrix representing the
objective and the constraints. Additionally, our program adds slack variables
(for example, (we change the inequalities in the constraints to equations---$3x
+ 2y \leq 5$ becomes $3x + 2y + s = 5$, where $s \geq 0$). This makes the
problem easier to solve). 

The simplex algorithm basically takes the objective function and attempts to
find a ``basic solution'' which will satisfy all the constraints. Even if this
solution is trivial (like setting all the variables in the objective to 0), we
take this solution and keep optimizing it by replacing a variable not in the
objective function with a variable that is. We check the objective to make sure
none of the coefficients are negative; if they are, we are finished and we have
the optimum value in the top right corner of the matrix (because we can set all
the variables in the objective to 0 and not violate the constraints). Our
``simple'' case is the case where we want to minimize the objective and all the
constraints are $\leq$ relations. However, we can change any problem to this
form. Maximization can be achieved by minimizing the negative of the objective.
Similarly, a $\geq$ constraint can be changed to a $\leq$ constraint by
multiplying each side of the inequality by $-1$. Finally, an $=$ constraint can
be changed into a $\geq$ and a $\leq$.

As to the design and implementation of our actual code base, it definitely can
be improved. It is the best it can be given the fact that we hadn't already done
this same problem before, but as with anything, it can be better. For example,
our \verb@initialize_simplex@ function ended up being very complex and
complicated. We all understand the general idea of what it does, but not all of
us could implement it by ourselves from scratch. Basically,
\verb@initialize_simplex@ takes in a system and adds the slack variables to it.  
During this step, if the algorithm determines a system (the objective
function and the constraints) to be unfeasible, it adds an extra variable and
attempts to minimize that (ie, replaces the objective function). If this new
system has a minimum of 0, then the original system is feasible and all we need
to do is substitute the old objective back in appropriately. Otherwise, the
original system is unfeasible and the program returns \verb@None@. In the first
case, an extra pivot operation might be necessary, but again,
\verb@initialize_simplex@ takes care of this. In the end, it calls
\verb@simple_solve@ (the function we originally wrote) on this now simple case
and a solution is produced. However, we did run our simplex algorithm against an
equation with 100 variables and 100 constraints (\verb@test22.txt@), and it
solved it in under 1 second. Even though the worst case running time of simplex
is not polynomial (it's exponential), on most real world cases simplex runs very
quickly, as evidenced by our test. 

Each member of the team contributed to the project. The work was split between
the matrix and elts modules, parsing the data, and the simplex algorithm. Andy
and Luis implemented most of the matrix module and parsing function, along with
the general framework of the program that allowed us to efficiently run and test
our code. Luis, Jason, and Ding worked on implementing and debugging the simplex
algorithm. Ding focused on comprehending every step of the algorithm as
presented in the Algorithms book and then presenting that information in a more
comprehensive manner to the group, while Jason focused on understanding the
process behind simplex and on coming up with example problems by hand. Jason and
Ding worked on testing simplex, and Luis and Andy provided much of the
functionality that allowed for easy testing. 

\section{Reflection}

Our original planning went pretty well and we accomplished most of what we
expected to accomplish. We mostly hit both our milestones, though we missed our
second milestone by a day. We were definitely surprised by how difficult simplex
ended up being, and there were times when we thought we were essentially done
with the algorithm and it turned out that we were only halfway there. 

Given more time, we would like to polish up input and outputs of our algorithm,
including a more sophisticated parsing function and a user friendly output.
Within the matrix module itself, there are a few functions that we would have
liked to implement including finding the eigenvectors/eigenvalues and
calculating the norm of a matrix, among others. 

We made the right choice when we decided to drop some of our unnecessary matrix
module functions to spend more time on simplex, but we made an early mistake of
not bothering to understand simplex until the latter half of the project, and
rather than understanding the whole algorithm before we started coding, we chose
to only understand the steps we were directly implementing, leading to the
restructuring of major portions of our code towards the end. The next time we
tackle an algorithm-centric project like this, we should definitely plan out the
entire algorithm before we code (including all the possible edge cases that
might pop up).

The most important thing we learned is how important good style, readability,
and careful planning is to programming and debugging. There were many times when
we were confused by what we wrote when we went back to debug our code, and since
no one was able to fully internalize every part of the algorithm, it was
important to abstract over helper functions and different parts of the code.
Having had someone who understood the algorithm entirely (or who had maybe
implemented something on this scale before), would have been extremely helpful
as it would have allowed us to better plan our time and resources as well as
abstract away many of the more detailed concepts.

\section{Advice for Future Students}

Start as early as possible and put some serious work in. Don't leave the video
until the last minute. This is a big project and showing it off is a very
rewarding part of it, especially when done well. Also try to understand the
general outline for every step before serious implementation, even if it means
delaying the implementation. Designing the framework is crucial and it makes the
whole process easier to get right the first time. Make sure to write test
functions as you write functions and to test each function right after you
finish writing it. This way, you will avoid having to hunt down a small bug in
one obscure part of your code.  You'll know exactly when and where your code
went wrong. 

Additionally, make sure your group members don't play too much League of Legends
or Starcraft!

But lastly---enjoy the project and be proud of what you've accomplished!

\end{document}
