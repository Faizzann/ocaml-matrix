\documentclass[letterpaper,11pt]{article}
% Change the header if you're going to change this.

% Possible packages - uncomment to use
%\usepackage{amsmath}       % Needed for math stuff
%\usepackage{lastpage}      % Finds last page
%\usepackage{amssymb}       % Needed for some math symbols
%\usepackage{graphicx}      % Needed for graphics
\usepackage[usenames,dvipsnames]{xcolor}        % Needed for graphics and color
%\usepackage{setspace}       % Needed to set single/double spacing
\usepackage{hyperref}

%Suggested by TeXworks
\usepackage[utf8]{inputenc} % set input encoding (not needed with XeLaTeX)
\usepackage{verbatim} % adds environment for commenting out blocks of text & for better verbatim
 
% Sets the page margins to 1 inch each side
\usepackage[margin=1in]{geometry}

% Uncomment this section if you wish to have a header.
\usepackage{fancyhdr} 
\pagestyle{fancy} 
\renewcommand{\headrulewidth}{0.5pt} % customise the layout... 
\lhead{CS51 Draft Spec} \chead{} \rhead{April 7, 2013} 
\lfoot{} \cfoot{\thepage} \rfoot{}

\frenchspacing
\setlength\parindent{0pt}
\setlength\parskip{2ex}

\newcommand{\annot}[1]{\textbf{\color{BrickRed} [#1]}}

\begin{document}
\title{CS 51 Final Project Draft Specification---\annot{Annotated!}}
\author{
Luis Perez $|$ luisperez@college.harvard.edu \\ 
Andy Shi $|$ andyshi@college.harvard.edu \\ 
Zihao Wang $|$ zihaowang01@college.harvard.edu \\ 
Ding Zhou $|$ dzhou@college.harvard.edu
}
\date{May 5, 2013}
\maketitle

%Put your stuff here
{\annot{Note for the reader: This is the annotated version of our CS 51 
final project specification. All annotations will appear in this color. Our changes to the actual modules and the finer points of implementation were too numerous to annotate (so there are no annotations in section 3)}}.

\section{Brief Overview}

Our project seeks to implement an efficient matrix library for Ocaml. The end
goal is to utilize this matrix library in conjunction with the simplex algorithm
to find solution to problems in linear programming. In order to accomplish this,
there will be multiple small checkpoints. We need to implement and efficient
matrix representation (so as to not use the naive implementation of lists of
lists). We also hope to provide a useful interface that allows the user to input
arbitrarily large matrices effectively (possible by mapping over the entries of
the matrix). One big challenge for the matrix library will be the implementation
of efficient algorithms for multiplication and row reduction. We expect the
multiplication to initially consists of the standard algorithm, but will
eventually grow into either block multiplication or Strassen {\annot{We were not
able to implement more sophisticated multiplication algorithms}}. We might even
implement some way to chose between the algorithms depending on the inputted
matrices, but that would not an essential feature {\annot{We were not able to do
this either}}.

The row reduction algorithm will certainly be one of the most fundamental
algorithms in our matrix library implementation. From our current understanding
of simplex, it seems like it will heavily rely on row reduction {\annot{The
simplex algorithm ended up relying on a different type of row reduction, so we
couldn't just plug in a matrix into our row reduction algorithm. We had to do a
lot of extra work to get simplex}}. The only algorithm we can come up with so
far is Gauss-Jordan, though we have found some algorithms which are more suited
for computation by computers with imprecise storage of floats {\annot{In fact,
we just had to be careful which element to choose as a pivot}}.

The simplex part of the problem is the main goal, but if we can get down the
matrix module, then it appears as if the simplex algorithm should be
straightforward {\annot{This was definitely not the case}}. Eventually, we wish
to implement a simplex module that allows the input of arbitrary linear
equations with linear constraints, and then we parse this information into the
corresponding matrix and finally perform the simplex algorithm on that matrix in
order to return a solution to the specified problem.

Other tentative goals are to implement matrices that can accept either
arbitrarily large numbers (bignums) or arbitrarily precise floats 
(\verb@int list * int stream@). It would also be interesting, though not
necessary, to implement multiple algorithms for each operation and choose from
them the one that is the best for a specific input data {\annot{We were not able
to implement any of this functionality}}.

\section{Feature List}
\begin{enumerate}

\item Find a way to represent a matrix (arrays or list list or other
representation)---will likely use an array representation using the Ocaml array
module (\url{http://caml.inria.fr/pub/docs/manual-ocaml/libref/Array.html})
{\annot{Implemented as specified}}

\item Make a matrix from a list of lists (this is how the user will input the
data of the matrix). {\annot{Implemented as specified}}

\item Scalar multiplication---as described here
(\url{http://www.purplemath.com/modules/mtrxmult.htm})
{\annot{Implemented as specified}}

\item Matrix addition---as described here
(\url{http://www.purplemath.com/modules/mtrxmult.htm})
{\annot{Implemented as specified}}

\item Standard Matrix multiplication
(\url{http://www.purplemath.com/modules/mtrxmult.htm}) 
{\annot{Implemented as specified}}

\item Row Reduction (Gauss-Jordan). We will use the following resource to help
us calculate row reduction when the entries of the matrix are stored with
limited precision
(\url{http://thejuniverse.org/PUBLIC/LinearAlgebra/LOLA/rowRed/var.html})
{\annot{Implemented as specified}}

\item Matrix inverse / Row reduction with LU Decomposition - Description can be
found here (\url{http://www.math.ust.hk/~macheng/math111/LU_Decomposition.pdf})
{\annot{To find the inverse, we augmented the original matrix with an identity
matrix, then row-reduced the augmented matrix}}

\item Simplex Algorithm (Described in Algorithms book, in addition to the
resources provided below)
{\annot{Implemented mostly as specified, with some modifications from the
Wikipedia article on simplex}}

\item Matrix norm 
{\annot{Did not implement}}

\item Matrix transpose
{\annot{Implemented as specified}}

\item Trace---This is a typical trace (sum of the entries on the diagonal)
{\annot{Implemented as specified}}

\item Determinant---found using algorithm described in Section 4.8 of Hubbard's
\emph{Vector Calculus, Linear Algebra, and Differential Forms: A Unified
Approach} (Math 23a/b textbook)
{\annot{Implemented by decomposing the matrix into upper and lower triangular
matrices, then multiplying the entries along the diagonals for both and
multiplying the two results}}

\item Eigenvectors / Eigenvalues, calculated also by a method in Hubbard's book
(this might not be necessary to carry out the simplex algorithm and we might
abandon it if we can't figure out how to implement it in a timely manner) 
{\annot{Did not implement}}

\end{enumerate}

\section{Draft Technical Specification}

\subsection{Matrices}

Our matrix module will actually be a functor that takes in an
\verb@ORDERED_AND_OPERATIONAL@ module (from ps4 and moogle) so we can have
matrices of ints, floats, or floats as streams. 

The following type definition (a la ps4) will give us a way to talk about order
for different data types.

\begin{verbatim}
type order = Equal | Less | Greater
\end{verbatim}

The following signature will allow us to provide matrix functionality on a
variety of data types. For example, we could make an
\verb@ORDERED_AND_OPERATIONAL@ type for floats, where add would be defined as
(+.), for example. In our matrix functor, we will then use the functions defined
in the signature below (for example, add instead of (+.)). Also, we have
included some generating functions to help us with testing. 

\begin{verbatim}
module type ORDERED_AND_OPERATIONAL
sig
  type t

  val zero : t
  val one: t
  val compare : t -> t -> order

  (* Converts a t to a string *)
  val to_string : t -> string

  val add: t -> t -> t
  val subtract: t -> t -> t
  val multiply: t -> t -> t
  val divide: t -> t -> t

  (* For testing *)
  (* Prints a t *)
  val print: t -> unit 

  (* Generates the same t each time when called *)
  val generate: unit -> t

  (* Generates a t greater than the argument passed in *)
  val generate_gt: t -> unit -> t

  (* Generates a t less than the argument passed in *)
  val generate_lt: t -> unit -> t

  (* Generates a t in between the two arguments. Returns none if none exist *)
  val generate_between: t -> t -> unit -> t option
end
\end{verbatim}

The following signature will provide all the necessary functions that can be
performed on a matrix. We include basic matrix operations like additions,
multiplications, scalar multiplication, and then standard operations like
finding the inverse of a matrix, finding its trace, determinant, eigenvalues,
transpose, norm, and how to row reduce the matrix. 

Below each function, in comments, is a brief description of how we will
implement the function. 

\begin{verbatim}
module type MATRIX = 
sig
  exception NonSquare
  
  type elt
  type matrix
  (* Type of this is unknown, but will probably be represented using Ocaml’s
   * built-in Arrays *)
  (* empty matrix *)
  val empty

  (* Takes a list of lists and converts that to a matrix *)
  val from_list : (elt list list) -> matrix
  (* Will implement using nested match statements *)

  (* Scales every element in the matrix by another elt *)
  val scale : matrix -> elt -> matrix
  (* Will implement by iterating through the matrix and scaling each element *)

  (* Adds two matrices. They must have the same dimensions *)
  val add : matrix -> matrix -> matrix
  (* Will add the elements elementwise and construct a new matrix *)

  (* Multiplies two matrices. If the matrices have dimensions m x n and p x q, n
   * and p must be equal, and the resulting matrix will have dimension m x q *)
  val mult: matrix -> matrix -> matrix
  (* Will take the dot product of the nth row of the first matrix and the jth
   * column of the second matrix to create the n,j th entry of the resultant *)

  (* Returns the row reduced form of a matrix *)
  val row_reduce: matrix -> matrix 
  (* We will implement the algorithm found in the link above *)

  (* Returns the inverse of a matrix *)
  val inverse: matrix -> matrix
  (* Will implement this based on the specification in the Algorithms book *)

  (* Returns the norm of the matrix *)
  val norm: matrix -> elt

  (* Transposes a matrix. If the input has dimensions m x n, the output will
   * have dimensions n x m *)
  val transpose: matrix -> matrix
  (* Will basically ``flip’’ the indices of the input matrix

  (* Returns the trace of the matrix *)
  val trace: matrix -> elt
  (* Will check if the matrix is square, then sum up all the elements along its
   * diagonal *)

  (* Returns the determinant of the matrix *)
  val det: matrix -> elt
  (* Will implement this algorithm based on a description in Hubbard. Involves
   * column reducing the input (or row-reducing the transpose) and then keeping 
   * track of the operations to build a sequence of coefficients to multiply *)

  (* Returns a list of eigenvalues and eigenvectors of a matrix *)
  val eigen: matrix -> (elt *matrix) list option
  (* Calculates successive powers of the input matrix, each multiplied by the
   * same basis vector. Generates a polynomial and solves for zeros, which 
   * yields eigenvalues. Repeat for all basis vectors *)

  (* Takes a string and builds a matrix from it *)
  from_string : string -> matrix
  (* We will have some way to express matrices using strings, and then we will
   * parse the string to give the matrix *)
  
  (* Prints out the contents of a matrix *)   
  print :matrix -> unit
  (* Iterate through the matrix and print each element *)
end
\end{verbatim}

The exception exists in case the user tries to perform operations, such as trace
or determinant, which require a square matrix. Matrices have their own type, and
they contain elements of the type \verb@elt@. 

We will probably need a helper function to raise matrices to powers and to solve
for zeros of a polynomial if we want to implement the \verb@eigen@ function. 

\subsection{Simplex Algorithm}

The simplex solves linear programs, which are basically linear inequalities.
Given inequalities, using the simplex algorithm, we can find the maximum value
of a given expression. What the simplex algorithm does is that it takes in
several linear inequalities and writes them so that they are equalities. For
instance $5x + 7y \leq 10$ would be $5x + 7y + s = 10$ where $s \geq 0$ is the
slack variable. Then it takes all these equations and puts them a a matrix where
each entry represents the coefficient of the variable in the expression. The
first row represents the equation we are optimizing. The final column is the
constant values of each equation. Then we will do a series of computations so
that the top row has all nonnegative values. This can be done using the
functions from the Matrix module. 

The steps are as follows:

Look along the first row and find the smallest negative element. Look at the
column of that entry and compute the value/element of each row. If any element
is zero, we can ignore it. Otherwise take the element with the smallest
value/element and make this the pivot point. Reduce the column so that the pivot
point in 1 and any other entry in the column is zero. Repeat the process again
until the top row consists of only nonnegative values. The first entry in the
last column is the maximum. In addition, for any column consisting of all zeroes
and a one, we can find the value of the corresponding variable by looking at the
value of the row containing the one. If the column doesn’t consists of zeroes
and a one, then the value of the corresponding variable is zero. 

This algorithm also works with variables that are negative, inequalities where
the unknowns is greater than a constant, and minimum of the equation. This can
be easily done by either multiplying the whole inequality by -1 or by negating
the variable. If calculating the minimum or negating the variable, we can revert
them later.  

\begin{verbatim}
module type simplex
sig
  type elt
  type linear_equation
  type constraint

  val pivot : matrix -> matrix
  val constraint_from_string : string -> constraint
  val linear_from_string : string -> constraint
end
\end{verbatim}

These are helper functions that we think will be used:

\verb@no_neg : matrix -> bool@ 

Check the top row has no negative elements, which can be done by using the next
helper function

\verb@min_of_row : row -> elt*(elt ref)@

A helper for the row reduction which will find the minimum element and its
location.

\section{What's Next}

We will read more into the algorithms and get a basic understanding of how they
all work so we can better judge the difficulty of each algorithm. We will also
read more into OCaml to see how to put together a project with multiple files
(since we have always relied on staff’s frameworks for each pset). We will
working through the CS50 appliance. We already have setup a git repository on
Github (\url{https://github.com/Fantastic-four}), and most of us are fairly
familiar with version control. 

\end{document}
